%% Pour compiler, on utilise XeLateX avec Ctrl-Shift-F1
%% �a produit une image PNG

\documentclass[convert={ghostscript, density = 1000}]{standalone}

\usepackage{pstricks-add}
\usepackage{pst-solides3d}
\usepackage{fp}

\begin{document}

\psset{unit=1,algebraic}
	\begin{pspicture}(-0.5,-1)(4.75,5)
		
		\psaxes[labels=none,ticks=none]{->}(0,0)(-0.5,-0.5)(4.75,4.75)
		
		\pswedge[fillstyle=solid,fillcolor=lightgray](0,0){3}{60}{70}
		\pswedge[fillstyle=solid,fillcolor=white](0,0){2.5}{60}{70}
		
		\multido{\i=1+1}{8}{%
			\FPeval{Ray}{\i/2}
			\psarc(0,0){\Ray}{10}{80}
		}
		\multido{\i=10+10}{8}{%
			\FPmul{\angle}{\i}{3.1415}
			\FPdiv{\angle}{\angle}{180}
			\FPcos{\x}{\angle}
			\FPsin{\y}{\angle}
			\FPmul{\x}{\x}{4.5}
			\FPmul{\y}{\y}{4.5}
			\psline(0,0)(\x,\y)
		}
		
		\psdot[dotsize=0.15](1.16,2.49)
		
		\uput[-90](2.46,-0.25){$r_{i-1}$}
		\psline{->}(2.46,-0.25)(2.46,0.43)
		\uput[-90](2.95,0.52){$r_{i}$}
		\uput[60](2.25,3.9){$\theta_{j-1}$}
		\uput[70](1.54,4.23){$\theta_{j}$}
		\psline{->}(1.16,2.49)(3.21,3.83)
		\uput[0](3.21,3.83){$(r^*_i,\theta^*_j)$}
		
	\end{pspicture}
\end{document}