%% Pour compiler, on utilise XeLateX avec Ctrl-Shift-F1
%% �a produit une image PNG

\documentclass[convert={ghostscript, density = 1000}]{standalone}

\usepackage{pstricks-add}

\begin{document}

\psset{algebraic,unit=2}
\begin{pspicture}(-3.25,-3.25)(3.25,3.25)	

\pscustom{
\parametricplot[linewidth=1.pt,plotpoints=200]{0.52359}{2.61799}{(3*sin(t))*cos(t)|(3*sin(t))*sin(t)}
\parametricplot[linewidth=1.pt,plotpoints=200]{2.61799}{0.52359}{(1+sin(t))*cos(t)|(1+sin(t))*sin(t)}
\fill[fillstyle=solid, fillcolor=lightgray]
}
\parametricplot[linewidth=1.5pt,plotpoints=200,linecolor=blue]{0}{6.28}{(3*sin(t))*cos(t)|(3*sin(t))*sin(t)}
\parametricplot[linewidth=1.5pt,plotpoints=200,linecolor=blue]{0}{6.28}{(1+sin(t))*cos(t)|(1+sin(t))*sin(t)}

\psaxes[labels=none,ticks=none]{->}(0,0)(-3.25,-3.25)(3.25,3.25)
\pscircle[linestyle=dotted](0,0){1}
\pscircle[linestyle=dotted](0,0){2}
\pscircle[linestyle=dotted](0,0){3}



\Polar
\psline[linestyle=dotted](-3,0)(3,0)
\psline[linestyle=dotted](-3,30)(3,30)
\psline[linestyle=dotted](-3,45)(3,45)
\psline[linestyle=dotted](-3,60)(3,60)
\psline[linestyle=dotted](-3,-30)(3,-30)
\psline[linestyle=dotted](-3,-45)(3,-45)
\psline[linestyle=dotted](-3,-60)(3,-60)
	
\end{pspicture}

\end{document}