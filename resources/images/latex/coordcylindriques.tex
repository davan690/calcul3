%% Pour compiler, on utilise XeLateX avec Ctrl-Shift-F1
%% �a produit une image PNG

\documentclass[convert={ghostscript, density = 1000}]{standalone}

\usepackage{pstricks-add}
\usepackage{pst-solides3d}
\usepackage{fp}
\def\nbpi{3.14159265}

\begin{document}

\begin{pspicture}(-3,-2)(5,5)
\psset{viewpoint=35 20 20 rtp2xyz,dotscale=2}
%viewpoint=4 2 2,
\psPoint(2,1.5,2){I}
\psPoint(2,0,0){x0}
\psPoint(0,1.5,0){y0}
\psPoint(2,1.5,0){xy}
\psPoint(1,0.75,0){xy0}
\psPoint(0,0,0){O}
\psLineIIID(2,1.5,0)(2,1.5,2)
\psLineIIID(0,0,0)(2,1.5,0)
\axesIIID(-0.5,-0.5,-0.5)(3,3,3)
\psdot(I)
\psdot(xy)
\uput{0.25}[r](I){$P=(r,\theta,z)$}
\uput{0.25}[r](xy){$(r,\theta,0)$}
\uput[45](xy0){$r$}
\psarc(O){1}{225}{310}
\psPoint(1.8,0.5,0){P}
\uput[-90](P){$\theta$}
\end{pspicture}

\end{document}