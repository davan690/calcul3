%% Pour compiler, on utilise XeLateX avec Ctrl-Shift-F1
%% �a produit une image PNG

\documentclass[convert={ghostscript, density = 1000}]{standalone}

\usepackage{pstricks-add}
\usepackage{pst-solides3d}
\usepackage{fp}

\begin{document}

\psset{unit=1,algebraic}
	\begin{pspicture}(-0.5,-1)(4.75,5)
		
		\pscustom{
			\pswedge(0,0){3.5}{10}{80}
			\fill[fillstyle=solid, fillcolor=lightgray]
		}
		\pscustom{
			\pswedge(0,0){1.5}{10}{80}
			\fill[fillstyle=solid, fillcolor=white]
		}
		
		\psaxes[labels=none,ticks=none]{->}(0,0)(-0.5,-0.5)(4.75,4.5)
		
		\uput[45](1.41,1.41){$D$}
		\uput[45](2.47,2.47){$r=b$}
		\uput[0](0.05,0.6){$r=a$}
		\uput[10](3.45,0.61){$\theta=\alpha$}
		\uput[80](0.61,3.45){$\theta=\beta$}
		
	\end{pspicture}
\end{document}