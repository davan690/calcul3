%% Pour compiler, on utilise XeLateX avec Ctrl-Shift-F1
%% �a produit une image PNG

\documentclass[convert={ghostscript, density = 1000}]{standalone}

\usepackage{pstricks-add}
\usepackage{pst-solides3d}
\usepackage{fp}

\begin{document}

\psset{unit=1,algebraic}
	\begin{pspicture}(-1,-1)(8.5,5.5)
		
		\psaxes[labels=none,ticks=none]{->}(0,0)(-1,-1)(8,5)[$x$,-90][$y$,180]
		
		\pscustom{
			\pscurve(1,3)(2,3.5)(3,3.8)(4,4.2)(5,3.5)(6,2.9)(7,2.4)
			\gsave
			\psline(7,2.4)(7,1)
			\pscurve(7,1)(6,1.5)(5,0.9)(4,0.7)(3,1)(2,1.2)(1,0.9)
			\psline(1,0.9)(1,3)
			\fill[fillstyle=solid, fillcolor=lightgray]
			\grestore
		}
		\pscurve[linewidth=1.5pt](1,3)(2,3.5)(3,3.8)(4,4.2)(5,3.5)(6,2.9)(7,2.4)
		\pscurve[linewidth=1.5pt](7,1)(6,1.5)(5,0.9)(4,0.7)(3,1)(2,1.2)(1,0.9)
		
		\psline[linestyle=dashed](7,2.4)(7,0) \uput[-90](7,0){$b$}
		\psline[linestyle=dashed](1,3)(1,0) \uput[-90](1,0){$a$}
		
		\psline(4,0)(4,5) \uput[-90](4,0){$x$}
		
		\uput[0](2.25,2.25){$D$}
		
		\uput[-90](6,1.5){$y_1(x)$}
		\uput{0.5}[90](6,2.9){$y_2(x)$}
		
	\end{pspicture}
\end{document}