%% Pour compiler, on utilise XeLateX avec Ctrl-Shift-F1
%% �a produit une image PNG

\documentclass[convert={ghostscript, density = 1000}]{standalone}

\usepackage{pstricks-add}
\usepackage{pst-solides3d}
\usepackage{fp}

\begin{document}

\psset{viewpoint=60 25 30 rtp2xyz,unit=1,algebraic,lightsrc=viewpoint}
	\begin{pspicture}(-3.5,-5)(7,3)
		
		\psPoint(2,0,0){P0}
		\psPoint(2,2,0){P1}
		\psPoint(0,2,0){P2}
		\psline[linestyle=dashed](P0)(P1)(P2)
		\uput[180](P0){$a$}
		\uput[90](P2){$c$}
		
		\psPoint(7,0,0){P3}
		\psPoint(7,2,0){P4}
		\psline[linestyle=dashed](P3)(P4)
		\uput[180](P3){$b$}
		
		\psPoint(0,7.25,0){P5}
		\psPoint(2,7.25,0){P6}
		\psline[linestyle=dashed](P5)(P6)
		\uput[90](P5){$d$}

		\axesIIID(0,0,0)(8,8,3)

		\multido{\nbA=2.5+0.5}{10}{%
			\multido{\nbB=2.5+0.5}{10}{%
				\FPeval{nbC}{(\nbA-0.5)*(\nbA-0.5)}
				\FPeval{nbD}{(\nbB-7)*(\nbB-7)}
				\FPeval{nbE}{\nbC+\nbD}
				\FProot{\nbE}{\nbE}{3}
				\FPexp{\nbF}{-\nbE}
				\FPeval{nbF}{\nbF*15}
				\FPeval{nbG}{\nbF/2}
				\psSolid[%
					object=parallelepiped,%
					a=0.5,b=0.5,c=\nbF,%
					fillcolor=yellow]%
					(\nbA,\nbB,\nbG)%
			}%
		}
		
	\end{pspicture}
\end{document}